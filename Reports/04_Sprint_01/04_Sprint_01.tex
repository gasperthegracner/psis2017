\documentclass[a4paper,11pt]{article}

\usepackage[slovene]{babel}
\usepackage[utf8]{inputenc}

\usepackage{listings}
\usepackage{babelbib}
\usepackage{url}

\usepackage{graphicx}
\usepackage{float}
\graphicspath{ {images/} }
\usepackage[usenames, dvipsnames]{color}

\usepackage{underscore}
\renewcommand{\lstlistingname}{Primer}% Listing -> Primer

\lstset{
numbers=left, 
numberstyle=\small, 
numbersep=8pt, 
frame = single, 
language=Python, 
framexleftmargin=15pt}

\setlength{\parindent}{0pt}
%BORDERS
\usepackage{geometry}
 \geometry{
 a4paper,
 total={170mm,257mm},
 left=30mm,
 right=25mm,
 top=30mm,
 bottom=30mm
 }

\begin{document}
\begin{titlepage}


% ZAČETNA STRAN
\newcommand{\HRule}{\rule{\linewidth}{0.5mm}} % Defines a new command for the horizontal lines, change thickness here

\center % Center everything on the page
 
%----------------------------------------------------------------------------------------
%	HEADING SECTIONS
%----------------------------------------------------------------------------------------

\textsc{ UNIVERZA V MARIBORU\\ FAKULTETA ZA ELEKTROTEHNIKO,\\RAČUNALNIŠTVO IN INFORMATIKO}\\[5cm] % Name of your university/college

%----------------------------------------------------------------------------------------
%	TITLE SECTION
%----------------------------------------------------------------------------------------
{ \huge \bfseries \textbf{POROČILO SPRINTA 1}}\\[0.4cm] % Title of your document
\textsc{\large Povezljivi sistemi in inteligentne storitve}\\[5cm] % Minor heading such as course title

%----------------------------------------------------------------------------------------
%	AUTHORS SECTION
%----------------------------------------------------------------------------------------
{\large Gašper Gračner}\\[0.4cm]
{\large Martin Oprešnik}\\[0.4cm]
{\large Luka Koštomaj}\\[0.4cm] 

%----------------------------------------------------------------------------------------
%	DATE SECTION
%----------------------------------------------------------------------------------------
\vfill % Fill the rest of the page with whitespace
{\large Maribor, Marec 2016}\\[3cm] % Date, change the \today to a set date if you want to be precise
\end{titlepage}
\newpage

%----------------------------------------------------------------------------------------
%	CONTENT SECTION
%----------------------------------------------------------------------------------------

\section{Predvidene naloge}
V prvem sprintu želimo detaljno analizirati podatke in jih pripraviti za nadaljno obdelavo zato planiramo naslednje naloge.
	\begin{enumerate}
		\item{Določitev strukture podatkov - \textcolor{OliveGreen}{DONE},}
		\item{Implematacija parserja - \textcolor{OliveGreen}{DONE},}
		\item{Določitev tipov analiz - \textcolor{OliveGreen}{DONE},}
		\item{Analiza podatkov - \textcolor{OliveGreen}{DONE},}
		\item{Izdelava učne in testne množice - \textcolor{OliveGreen}{DONE}}
	\end{enumerate}
\newpage

\section{Implematacija parserja - Martin Oprešnik}
Podatki so podani v formatu gpx, ki ni najbolj primeren za nadaljno obdelavo. Zato smo naredili razčlenjevalnik, ki podatke pretvori, agregira in shrani v csv format. Ta najprej prebere gpx datoteke vsakega atleta. Iz ene gpx datoteke, oz. enega treninga nastane ena vrstica v csv datoteki. Tu shranjujemo naslednje podatke: trajanje, povprečni in maksimalni srčni utrip, datum, višinsko razliko, razdaljo v 2D in 3D prostoru, čas gibanja in čas mirovanja. Ko to naredimo za vse atlete, se podatki združijo v eno datoteko.

\newpage
\section{Analiza podatkov - Gašper Gračner}
Najprej smo želeli izvedeti količino podatkov glede na športnike, ki jih imamo v našem datasetu, saj nas je zanimo s kakšnimi količinami podatkov za posameznega športnika imamo opravka. \\

\begin{figure}[H]
\caption{Število podatkov glede na športnike}
\centering
\includegraphics[width=1\textwidth]{ridersTrainingsCount}
\end{figure}

Ker je bila naša naloga razdeliti intenzivnost treninga glede na srčni utrip smo nad podatki povprečnega srčnega utripa zagnali gručenje. S tem postopkom smo dobili tri nivoje intenzivnosti. Za gručenje (K-means) smo se odločili, zaradi pouporabnosti postopka, saj lahko v primeru, da želimo več nivojev le povečamo število K.\\

\begin{figure}[H]
\caption{Gručenje srčnega utripa}
\centering
\includegraphics[width=1\textwidth]{HrCLusters}
\end{figure}
 
Rezultat spodaj prikazanega procesa je gručenje in dve .CSV datoteki. Ena z vsemi atributi in druga z samo izbranimi atributi
\begin{enumerate}
\item{Povprečen srčni utrip,}
\item{intenzivnost vadbe,}
\item{id športnika,}
\item{trajanje vadbe}
\item{datum treninga}
\end{enumerate} 

Te podatke smo nato obdelali, tako da smo združili treninge posameznega športnika po tednih. Za vsak dan v tednu smo določili intenzivnost in trajanje treninga. Podatki v tej obliki so pripravljeni za klasifikacijo.

\begin{figure}[H]
\caption{Proces analize podatkov}
\centering
\includegraphics[width=1\textwidth]{IntensityClusteringProcess}
\end{figure}


\newpage
\section{Izdelava učne in testne množice - Luka Koštomaj}
Za izdelavo učne in testne množice smo uporabili skiptni jezik R.
Tukaj smo lahko na zelo enostaven način obdelali podatke.
Ustvarili smo tri različne testne in učne množice.
Prvo učno množico sestavljajo vsi nadpovprečni atleti, testna pa je sestavljena iz preostalih podatkov.
Druga učna in testna množica je sestavljena iz random podatkov.
Tretja učna množica pa je sestavljena boljših atletov ampak manjkajo trije atleti, ki so bili filtrirani. 
Testna množica pa vsebuje tudi izpuščene atlete.

\begin{figure}[H]
\caption{graf prikazuje srčni utrip v povezavi z trajanjem treninga}
\centering
\includegraphics[width=1\textwidth]{athleteClassification}
\end{figure}

\end{document}
