\documentclass[a4paper,11pt]{article}

\usepackage[slovene]{babel}
\usepackage[utf8]{inputenc}

\usepackage{listings}
\usepackage{babelbib}
\usepackage{url}

\usepackage{graphicx}
\graphicspath{ {images/} }
\usepackage[usenames, dvipsnames]{color}

\usepackage{underscore}
\renewcommand{\lstlistingname}{Primer}% Listing -> Primer

\usepackage{array}
\usepackage{tabu}

\lstset{
numbers=left, 
numberstyle=\small, 
numbersep=8pt, 
frame = single, 
language=Python, 
framexleftmargin=15pt}

\setlength{\parindent}{0pt}
%BORDERS
\usepackage{geometry}
 \geometry{
 a4paper,
 total={170mm,257mm},
 left=30mm,
 right=25mm,
 top=30mm,
 bottom=30mm
 }

\begin{document}
\begin{titlepage}


% ZAČETNA STRAN
\newcommand{\HRule}{\rule{\linewidth}{0.5mm}} % Defines a new command for the horizontal lines, change thickness here

\center % Center everything on the page
 
%----------------------------------------------------------------------------------------
%	HEADING SECTIONS
%----------------------------------------------------------------------------------------

\textsc{ UNIVERZA V MARIBORU\\ FAKULTETA ZA ELEKTROTEHNIKO,\\RAČUNALNIŠTVO IN INFORMATIKO}\\[5cm] % Name of your university/college

%----------------------------------------------------------------------------------------
%	TITLE SECTION
%----------------------------------------------------------------------------------------
{ \huge \bfseries \textbf{Sprint 3}}\\[0.4cm] % Title of your document
\textsc{\large Povezljivi sistemi in inteligentne storitve}\\[5cm] % Minor heading such as course title

%----------------------------------------------------------------------------------------
%	AUTHORS SECTION
%----------------------------------------------------------------------------------------
{\large Gašper Gračner}\\[0.4cm]
{\large Martin Oprešnik}\\[0.4cm]
{\large Luka Koštomaj}\\[0.4cm] 

%----------------------------------------------------------------------------------------
%	DATE SECTION
%----------------------------------------------------------------------------------------
\vfill % Fill the rest of the page with whitespace
{\large Maribor, April 2016}\\[3cm] % Date, change the \today to a set date if you want to be precise
\end{titlepage}
\newpage

%----------------------------------------------------------------------------------------
%	CONTENT SECTION
%----------------------------------------------------------------------------------------

\section{Predvidene naloge}
V drugem sprintu smo si vsi zadali analizo oz. testiranje izbranih metod strojnega učenja.
	\begin{enumerate}
		\item{Gašper Gračner- Optimizacija parametrov SVM - \textcolor{OliveGreen}{DONE},}
		\item{Luka Koštomaj - Izboljšava učenja nevronske mreže,}
		\item{Martin Oprešnik - Priporočanje treninga}
	\end{enumerate}
	
\newpage
\section{Gašper Gračner- Optimizacija parametrov SVM}
Po analizi parametrov se bom lotil same implementacije rešitve in prilaganja parametrov, za naš problem, saj predvidevam, da bom s prilagajanjem paramtrov lahko rahlo izboljšal začetno rešitev. Zasnoval pa si bom še tesno množico, v primeru, da tega ne bomo uspeli v prvem sprintu.\\

\begin{tabu} to 0.8\textwidth { || X[l] | X[c] || }
 \hline
 Parametri & Natančnost[\%] \\
 \hline
 \hline
 kernel="linear"  & 98.25   \\
 \hline
 kernel="poly"  & 97.00   \\
 \hline
 kernel="rbr"  & 76.00   \\
 \hline
 kernel="\ sigmoid"  & 4.75   \\
 \hline
 kernel="linear", max_iter=50000  & 98.5  \\
 \hline
 kernel="linear", decision_function_shape="ovr"  & 98.25  \\
 \hline
 kernel="linear", decision_function_shape="ovo"  & 98.25  \\
 \hline
 kernel="linear", shrinking=False & 98.25  \\
 \hline
 kernel="poly", max_iter=50000  & 97.0  \\
 \hline
 kernel="poly", decision_function_shape="ovr"  & 97.0  \\
 \hline
 kernel="poly", decision_function_shape="ovo"  & 97.0  \\
 \hline
 kernel="poly", shrinking=False & 97.0  \\
\hline
\end{tabu}

\

\section{Luka Koštomaj - Izboljšava učenja nevronske mreže}

V tretjem sprintu bom implementiral nevronsko mrežo oz. izboljšal učenje nevronske mreže.

\section{Martin Oprešnik - Priporočanje treninga }

V tretjem sprintu bom poskusil implementirati priporočanje treninga. Odločitveno drevo se bo naučilo kak trening je priporočen za podanega uporabnika. Te rezultate bom primerjal s testno množico. 




\end{document}