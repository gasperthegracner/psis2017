\documentclass[a4paper,11pt]{article}

\usepackage[slovene]{babel}
\usepackage[utf8]{inputenc}

\usepackage{listings}
\usepackage{babelbib}
\usepackage{url}

\usepackage{graphicx}
\usepackage{float}
\graphicspath{ {images/} }
\usepackage[usenames, dvipsnames]{color}

\usepackage{multicol}
\setlength{\columnsep}{1cm}

\usepackage{underscore}
\renewcommand{\lstlistingname}{Primer}% Listing -> Primer

\usepackage{amsmath}
\usepackage{bm}

\usepackage{array}
\usepackage{tabu}

\lstset{
numbers=left, 
numberstyle=\small, 
numbersep=8pt, 
frame = single, 
language=Python, 
framexleftmargin=15pt}

\setlength{\parindent}{0pt}
%BORDERS
\usepackage{geometry}
 \geometry{
 a4paper,
 total={170mm,257mm},
 left=30mm,
 right=25mm,
 top=30mm,
 bottom=30mm
 }

\begin{document}

\huge { Priporočanje športnega treninga na podlagi prejšnjih aktivnosti\\\\} 
\normalsize
[Summary]
\\

\begin{multicols}{2}

\section{Uvod}
Veliko ljudi se dandanes ukvarja z najrazličnejšimi športi, nekateri profesionalno, z namenom, da dosežejo čim boljše rezultate in osvojijo medalije na tekmovanjih, drugi pa se s športom ukvarjajo povsem rekreacijsko. Skoraj vsi športniki pa se borijo s samim sabo in želijo doseči nek cilj.\\
Kot glavni faktor, ki pripomore k izboljšavm lahko štejemo kritike, mnenja in predloge, ki jih dobimo od trenerjev oz. akterja, ki spremlja našo športno aktivnost. Trenutna tehnologija omogoča bogat nabor podatkov, ki jih lahko dobimo v času treningov in tekem. Veliko atletov in trenerjev se strinja, da do ti podatki neprecenljivega pomena\cite{Liebermann}. Velika večina ljudi ima danes ob sebi pametne telefone, nekateri celo pametne ure in razne športne ure, ki omogočajo zbiranje podatkov o aktivnosti osebe, natančnost podatkov pa je odvisna od tipa posamezne naprave in proizvajalca čipa\cite{Case}.\\
Pridobljeni podatki, nam brez prave obdelave in predstavitve ne prinašajo velike vrednosti, zato bomo v  članku opisali metode, s katerimi smo podatke obdelali in iz njih pridobili uporabne informacije. Naša naloga je bila preizkusiti različne metode strojnega učenja, ki bi trenerjem in rekreativnim športnikom pomagale pri priporočanju čim bolj optimalnega treninga.\\
V nadaljevanju članka, je so navedena in opisana povezana dela in že odstoječe aplikacije, ki obstajo na tem področju. Opisan je postopek priprave podatkov in uporabljene metode strojnega učenja. V drugem delu čanka pa je predstavljen eksperiment in rezultati našega dela.


\section{Povezana dela}
[Tukaj povezana dela]

\section{Priprava podatkov}
[Parser in podatki itd....]

\section{Metode strojnega učenja}

\subsection{Odločitveno drevo}


\subsection{Nevronske mreže}


\subsection{Metoda podpornih vektorjev}



\section{Predstavitev eksperimenta}

\subsubsection{Nabor podatkov}


\subsection{Eksperiment}


\subsection{Rezultati}


\subsection{Zaključek}


\end{multicols}

\normalsize
\newpage

\begin{multicols}{2}


\bibliography{mybib}{}
\bibliographystyle{plain}

\end{multicols}

\end{document}