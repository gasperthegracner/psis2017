\documentclass[a4paper,11pt]{article}

\usepackage[slovene]{babel}
\usepackage[utf8]{inputenc}

\begin{document}
\begin{titlepage}


% ZAČETNA STRAN
\newcommand{\HRule}{\rule{\linewidth}{0.5mm}} % Defines a new command for the horizontal lines, change thickness here

\center % Center everything on the page
 
%----------------------------------------------------------------------------------------
%	HEADING SECTIONS
%----------------------------------------------------------------------------------------

\textsc{ UNIVERZA V MARIBORU\\ FAKULTETA ZA ELEKTROTEHNIKO,\\RAČUNALNIŠTVO IN INFORMATIKO}\\[5cm] % Name of your university/college

%----------------------------------------------------------------------------------------
%	TITLE SECTION
%----------------------------------------------------------------------------------------
{ \huge \bfseries \textbf{DEFINICIJA PROBLEMA}}\\[0.4cm] % Title of your document
\textsc{\large Povezljivi sistemi in ineligentne storitve}\\[5cm] % Minor heading such as course title

%----------------------------------------------------------------------------------------
%	AUTHORS SECTION
%----------------------------------------------------------------------------------------
{\large Gašper Gračner}\\[0.4cm]
{\large Martin Oprešnik}\\[0.4cm]
{\large Luka Koštomaj}\\[0.4cm] 

%----------------------------------------------------------------------------------------
%	DATE SECTION
%----------------------------------------------------------------------------------------
\vfill % Fill the rest of the page with whitespace
{\large Maribor, avgust 2016}\\[3cm] % Date, change the \today to a set date if you want to be precise
\end{titlepage}
\newpage

%----------------------------------------------------------------------------------------
%	CONTENT SECTION
%----------------------------------------------------------------------------------------
\textbf{Izbrani projekt: }Projekt 3 – Priporočanje športnega treninga na podlagi prejšnjih aktivnosti \\

\textbf{Krajši opis: } Planiranje športnega treninga je dokaj težka naloga, za katero so zadolženi
trenerji. V današnji dobi računalnikov pričakujemo, da bodo v prihodnosti glavni
asistenti trenerjev vglavnem računalniki. Najnovejše raziskave trenutno razkrivajo, da
se za priporočanje optimalnega treninga največkrat uporabljajometode, kot npr.
podatkovno rudarjenje, strojno učenje in ostale.\\

\textbf{Zahteve: } Študenti dobijo arhiv športnih dejavnosti v obliki TCX ali GPX. Te
podatke je najprej potrebno obdelati in iz njih izluščiti atribute, kot so npr. čas trajanja,
število kilometrov, maksimalnioz. povprečni pulz med športno aktivnostjo, itd. Iz teh
podatkov je potrebno tvoriti testno in validacijsko množico, ter na teh dveh množicah
uporabiti eno izmed metod strojnega učenja glede na kriterije, ki jih določi študent. Ta
projekt lahko izvaja več študentov, pri čemer prvi recimo vzame metodo Decision
Trees, drugi Random Forest, in tretji Extreme Randomized Trees. Na koncu je treba
narediti poročilo kjer se predstavi rešitev in rezultati.

\end{document}