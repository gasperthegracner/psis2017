\documentclass[a4paper,11pt]{article}

\usepackage[slovene]{babel}
\usepackage[utf8]{inputenc}

\usepackage{listings}
\usepackage{babelbib}
\usepackage{url}

\usepackage{underscore}
\renewcommand{\lstlistingname}{Primer}% Listing -> Primer

\lstset{
numbers=left, 
numberstyle=\small, 
numbersep=8pt, 
frame = single, 
language=Python, 
framexleftmargin=15pt}

\setlength{\parindent}{0pt}
%BORDERS
\usepackage{geometry}
 \geometry{
 a4paper,
 total={170mm,257mm},
 left=30mm,
 right=25mm,
 top=30mm,
 bottom=30mm
 }

\begin{document}
\begin{titlepage}


% ZAČETNA STRAN
\newcommand{\HRule}{\rule{\linewidth}{0.5mm}} % Defines a new command for the horizontal lines, change thickness here

\center % Center everything on the page
 
%----------------------------------------------------------------------------------------
%	HEADING SECTIONS
%----------------------------------------------------------------------------------------

\textsc{ UNIVERZA V MARIBORU\\ FAKULTETA ZA ELEKTROTEHNIKO,\\RAČUNALNIŠTVO IN INFORMATIKO}\\[5cm] % Name of your university/college

%----------------------------------------------------------------------------------------
%	TITLE SECTION
%----------------------------------------------------------------------------------------
{ \huge \bfseries \textbf{PREGLED SORODNIH DEL IN NAČRT}}\\[0.4cm] % Title of your document
\textsc{\large Povezljivi sistemi in inteligentne storitve}\\[5cm] % Minor heading such as course title

%----------------------------------------------------------------------------------------
%	AUTHORS SECTION
%----------------------------------------------------------------------------------------
{\large Gašper Gračner}\\[0.4cm]
{\large Martin Oprešnik}\\[0.4cm]
{\large Luka Koštomaj}\\[0.4cm] 

%----------------------------------------------------------------------------------------
%	DATE SECTION
%----------------------------------------------------------------------------------------
\vfill % Fill the rest of the page with whitespace
{\large Maribor, avgust 2016}\\[3cm] % Date, change the \today to a set date if you want to be precise
\end{titlepage}
\newpage

%----------------------------------------------------------------------------------------
%	CONTENT SECTION
%----------------------------------------------------------------------------------------

\section{Pregled sorodnih del}
\section{Načrt dela}

Projektno delo smo si razdelili na tri dele in sicer tako, da bomo za priporočanje športne aktivnosti uporabili različne pristope strojnega učenja. 
Uporabili bomo neural networks, SVM in . 
Vsi se bomo prvič srečali z metodami stojnega učenja zato je načrtovanje dela kar zahtevno. \\
Pri delu bomo uporabili Python in njegove knjižnice, za sledenje delu pa GitHub.com. 

\subsection{Sprint1}
V prvem sprintu smo si zadali analizo podatkov, v kateri bomo sodlovali vsi. Podatke bomo brali s pomočjo Python knjižnice \textit{gpxpy (https://github.com/tkrajina/gpxpy)}. Potrebne funkcionalnosti knjižnice bomo enkapsulirali, da jih bomo lahko enostavneje uporabili pri nadaljnem delu.\\ Nad podatki bomo izvedli gručanje in jih prikazali v grafični obliki, da si bomo lažje predstavljali njihov pomen. Poskušali pa bomo ustvariti še učno množico podatkov.


\subsection{Sprint2}
\subsubsection{Gašper - Analiza parametrov in delovanja metode}
Med raziskovanjem o SVM sem naletel na opis te metode v programskem jeziku Python. Uporabil je knjižnico sklearn in modul svm. Zato bom v tem sprintu poskušal analizirati parametre \textit{cache_size}, \textit{class_weight}, \textit{coef0}, \textit{decision_function_shape}, \textit{degree}, \textit{gamma}, \textit{kernel}, \textit{max_iter}, \textit{probability}, \textit{random_state}, \textit{shrinking}, \textit{tol}, in \textit{verbose}. \\

\begin{lstlisting}[caption={Uporaba SVM v Pythonu},captionpos=b]
from sklearn import svm
X = [[0, 0], [1, 1]]
y = [0, 1]
clf = svm.SVC()
clf.fit(X, y)  
SVC(C=1.0, cache_size=200, class_weight=None, 
	coef0=0.0, decision_function_shape=None, 
	degree=3,
    gamma='auto', kernel='rbf',
    max_iter=-1, probability=False, 
    random_state=None, shrinking=True,
    tol=0.001, verbose=False)
\end{lstlisting}

\subsubsection{Luka - Analiza parametrov in izbira ustreznega modela nevronskih mrež}
Med raziskovanjem nevronskih mrež sem našel veliko različnih metod za izdelavo le teh. Zato se trenutno še ne morem odločiti kateri model bom uporabil. Omejil sem se na programski jezik Python in tri knjižnice. Knjižnice, ki sem jih izbral so: Theano, Caffe, TensorFlow. Možnost tudi obstaja, da ne bi uporabil nobeno od naštetih. Moral bom tudi analizirati vhodne in izhodne podatke, ki jih bom pridobil v prvem sprintu, da bom lahko zgradil model.

\subsection{Sprint3}
\subsubsection{Gašper}
Po analizi parametrov se bom lotil same implementacije rešitve in prilaganja parametrov, za naš problem, saj predvidevam, da bom s prilagajanjem paramtrov lahko rahlo izboljšal začetno rešitev. Zasnoval pa si bom še tesno množico, v primeru, da tega ne bomo uspeli v prvem sprintu.

\subsubsection{Luka}
V tretjem sprintu bom implementiral nevronsko mrežo.

\subsection{Sprint4}
\subsubsection{Gašper}
Na koncu, bom poskusal uporabiti in primerjati še različne klasifikatorje, ki podpirajo SVM in so implementirani v uporabljeni knjižnici (\textit{SVC}, \textit{NuSVC} in \textit{LinearSVC}). Moral pa bom določiti tudi primerno metodo primerjanja klasifikatorjev.

\subsubsection{Luka}
Na koncu, bom testiral nevronsko mrežo in s tem poskušal spremeniti testno množico, da bi mogoče izboljšalo napoved.

\newpage
\section{SEZNAM LITERATURE}{\baselineskip=-8pt}
\vspace{-36pt}
\renewcommand{\refname}{}
\nocite{*}
\bibliographystyle{unsrt}
\bibliography{bib/test}% database is "test.bib" located in a "bib" subfolder 

\end{document}