\documentclass[a4paper,11pt]{article}

\usepackage[slovene]{babel}
\usepackage[utf8]{inputenc}

\usepackage{listings}
\usepackage{babelbib}
\usepackage{url}

\usepackage{graphicx}
\graphicspath{ {images/} }

\usepackage{subfigure}

\usepackage{underscore}
\renewcommand{\lstlistingname}{Primer}% Listing -> Primer

\lstset{
numbers=left, 
numberstyle=\small, 
numbersep=8pt, 
frame = single, 
language=Python, 
framexleftmargin=15pt}

\setlength{\parindent}{0pt}
%BORDERS
\usepackage{geometry}
 \geometry{
 a4paper,
 total={170mm,257mm},
 left=30mm,
 right=25mm,
 top=30mm,
 bottom=30mm
 }

\begin{document}
\begin{titlepage}


% ZAČETNA STRAN
\newcommand{\HRule}{\rule{\linewidth}{0.5mm}} % Defines a new command for the horizontal lines, change thickness here

\center % Center everything on the page
 
%----------------------------------------------------------------------------------------
%	HEADING SECTIONS
%----------------------------------------------------------------------------------------

\textsc{ UNIVERZA V MARIBORU\\ FAKULTETA ZA ELEKTROTEHNIKO,\\RAČUNALNIŠTVO IN INFORMATIKO}\\[5cm] % Name of your university/college

%----------------------------------------------------------------------------------------
%	TITLE SECTION
%----------------------------------------------------------------------------------------
{ \huge \bfseries \textbf{Sprint 2}}\\[0.4cm] % Title of your document
\textsc{\large Povezljivi sistemi in inteligentne storitve}\\[5cm] % Minor heading such as course title

%----------------------------------------------------------------------------------------
%	AUTHORS SECTION
%----------------------------------------------------------------------------------------
{\large Gašper Gračner}\\[0.4cm]
{\large Martin Oprešnik}\\[0.4cm]
{\large Luka Koštomaj}\\[0.4cm] 

%----------------------------------------------------------------------------------------
%	DATE SECTION
%----------------------------------------------------------------------------------------
\vfill % Fill the rest of the page with whitespace
{\large Maribor, April 2016}\\[3cm] % Date, change the \today to a set date if you want to be precise
\end{titlepage}
\newpage

%----------------------------------------------------------------------------------------
%	CONTENT SECTION
%----------------------------------------------------------------------------------------

\section{Predvidene naloge}
V drugem sprintu smo si vsi zadali analizo oz. testiranje izbranih metod strojnega učenja.
	\begin{enumerate}
		\item{Analiza parametrov in delovanja metode SVM (Gašper),}
		\item{Analiza parametrov in izbira ustreznega modela nevronskih mrež (Luka),}
		\item{Analiza delovanja metode odločitveno drevo (Martin),}
		\item{Prilagoditev podatkov}
	\end{enumerate}
	

\newpage
\section{ SVM (Gašper Gračner)}

\subsection{Prednosti SVM}
\begin{enumerate}
	\item{Deluje v večdimenzionalnem prostoru,}
	\item{Efektiven je tudi v primerih, ko je število atributov večj kot število primerkov,}
	\item{Prostorsko učinkovit,}
	\item{Podpira različne funkcije jedra (lahko tudi uporabniške)}
\end{enumerate}

\subsection{Slabosti SVM}
\begin{enumerate}
	\item{Slabša učinkovitost v primeru da je število atributov veliko večje od števila primerkov,}
	\item{SVM-ji direktno ne določajo verjetnosti, zato je ta izračunana s pomočjo \textit{five-fold cross-validation}}
\end{enumerate}

\subsection{Analiza parametrov}

\subsubsection{kernel}
	Določa tip jedra (fukcijo za razmejitev), ki bo uporabljen pri algoritmu:
	\begin{itemize}
		\item{'linear',}
		\item{'poly',}
		\item{'rfb' (privzeto),}
		\item{'sigmoid',}
	\end{itemize}
	

\subsubsection{degree}
	Upošteva se le v primeru, da je izbran \textit(kernel='poly'), določa pa stopnjo polinoma.

\subsubsection{gamma}
	Koeficient jedra, v primeru da je nastavljen \textit{'auto'} je vrednost \textit{1/n_features}. Samo pri jedrih 'rfb', 'poly' in 'sigmoid'.
	
\subsubsection{coef0}
	Neodvisna vrednost jedrne funkcije. ('poly','sigmoid').

\subsubsection{probability}
	Če želimo da nam metoda izračuna še verjetnost. Privzeta vrednost je 'False'.

\subsubsection{shrinking}
	Skrčitvena hevristika, privzeta vrednost je \textit{'True'}.

\subsubsection{tol}
	Toleranca pri zaustavitvenem kriteriju, privzeta vrednost je \textit{1e-3}.
	
\subsubsection{cache_size}
	Velikost pomnilnika funkcije jedra v MB.

\subsubsection{class_weight}
	Obtežitev razdredov, če paramaeter ni podan so vsi razredi enakovredni.

\subsubsection{verbose}
	Omogočanje verbose izhoda.

\subsubsection{max_iter}
	Določanje maximalnega števila iteracij. Vrednost \textit{-1} pomeni, da število iteracij ni omejeno.

\subsubsection{decision_function_shape}
	Določitev izhoda odločitvene funkcije.
	\begin{itemize}
		\item{'ovo'  - (n_samples, n_classes * (n_classes - 1) / 2),}
		\item{'ovr'  - (n_samples, n_classes) }
		\item{None  - deprecated,}
	\end{itemize}

\subsubsection{random_state}
	Določitev zrna pri izbiri naključnih vrednosti.

\section{Nevronske mreže (Luka Koštomaj)}
V tem sprintu sem implementiral nevronske mreže s pomočjo knjižice scikit-learn.
Na testni množici je algroitem dosegel 70% natančnost.
Potrebno bo popraviti testno množico, da bomo lahko dosegli boljši rezultat.

\subsection{Analiza parametrov}
	Opisani so uporabljeni parametri, ki sem jih uporabil pri izdelavi nevronske mreže.

	\subsubsection{solver}
		poseben parameter, kjer nastavimo kateri algoritem se uporablja pri distribuciji teže
		lbfgs - quasi-Newton metoda,
		sgd - stochastic gradient descent,
		adam - stochastic gradient-based optimizer (Kingma, Diederik, Jimmy Ba).

	\subsubsection{hidden_layer_sizes}
		je število neuronov v skritem sloju.

	\subsubsection{activation}
		izbira funkcije
		identity - f(x) = x
		logistic - f(x) = 1 / (1 + exp(-x)).
		tanh - f(x) = tanh(x).
		relu - f(x) = max(0, x)
 
\section{Odločitveno drevo (Martin Oprešnik) }



\end{document}